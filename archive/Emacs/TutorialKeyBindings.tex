% Created 2018-10-21 Sun 19:29
% Intended LaTeX compiler: pdflatex
\documentclass[11pt]{article}
\usepackage[utf8]{inputenc}
\usepackage[T1]{fontenc}
\usepackage{graphicx}
\usepackage{grffile}
\usepackage{longtable}
\usepackage{wrapfig}
\usepackage{rotating}
\usepackage[normalem]{ulem}
\usepackage{amsmath}
\usepackage{textcomp}
\usepackage{amssymb}
\usepackage{capt-of}
\usepackage{hyperref}
\author{Pramod Nepal}
\date{\today}
\title{}
\hypersetup{
 pdfauthor={Pramod Nepal},
 pdftitle={},
 pdfkeywords={},
 pdfsubject={},
 pdfcreator={Emacs 26.1 (Org mode 9.1.9)},
 pdflang={English}}
\begin{document}

\tableofcontents

Note: Read PackageInstallation note first. After installation of swiper and projectile package follow this.\\


\section{Swiper}
\label{sec:orgdb1907a}
After installing swiper you can invoke swiper with\\
\begin{center}
\begin{tabular}{ll}
C-s & Search with swiper\\
\end{tabular}
\end{center}

\section{Jump between edits}
\label{sec:org94dde7b}
\begin{center}
\begin{tabular}{ll}
\hline
C-u C-space & jump to previous edit position\\
C-x C-space & if you want to navigate back between buffers\\
\end{tabular}
\end{center}

\section{Copy/Save}
\label{sec:org099327a}
\begin{center}
\begin{tabular}{ll}
\hline
M-w w & Save word at point\\
M-w s & Save sexp at point\\
M-w l & Save list at point including sexp\\
M-w d & Save defun\\
M-w f & Save file at point\\
M-w b & Save buffer\\
\end{tabular}
\end{center}

\section{Macro}
\label{sec:org5ce70e2}
\begin{center}
\begin{tabular}{ll}
\hline
C-x ( & Start defining a macro\\
C- ) & Stop defining a macro\\
C-e & Execute macro\\
C-u 37 C-x e & Execute 37 time\\
\end{tabular}
\end{center}

\section{Case}
\label{sec:org8417a26}
\begin{center}
\begin{tabular}{ll}
\hline
C-x C-u & uppercase\\
C-x C-l & lowercase\\
\end{tabular}
\end{center}

\section{Selection}
\label{sec:orga248c37}
\begin{center}
\begin{tabular}{ll}
\hline
C-x SPC & rectangle-mark-mode\\
\end{tabular}
\end{center}

\section{Customization}
\label{sec:orgd4e4a42}
\subsection{Change theme}
\label{sec:orgb3b12c4}
\begin{center}
\begin{tabular}{ll}
\hline
M-x & customize-themes\\
\end{tabular}
\end{center}

\section{Key-chords (Quick Jump in edit)}
\label{sec:org66937fc}
Key-chords are available only when the prelude-key-chord module has been enabled.\\

\begin{center}
\begin{tabular}{ll}
Keybinding & Description\\
\hline
jj & Jump to the beginning of a word(avy-goto-word-1)\\
jk & Jump to a character(avy-goto-char)\\
jl & Jump to the beginning of a line(avy-goto-line)\\
JJ & Jump back to previous buffer(crux-switch-to-previous-buffer)\\
uu & View edits as a tree(undo-tree-visualize)\\
xx & Executed extended command(execute-extended-command)\\
yy & Browse the kill ring(browse-kill-ring)\\
\end{tabular}
\end{center}

\begin{center}
\begin{tabular}{ll}
\hline
pp & Goto previous line\\
nn & Goto next line\\
bb & List bookmark\\
bm & Add bookmark\\
\end{tabular}
\end{center}

;; Custom keys defined in init-personalize.el\\
\begin{verbatim}
(key-chord-define-global "nn" 'next-line)
(key-chord-define-global "pp" 'previous-line)
(key-chord-define-global "bb" 'bookmark-bmenu-list)
(key-chord-define-global "bm" 'bookmark-set)
\end{verbatim}

\section{Convert Document}
\label{sec:orgace9259}
\begin{verbatim}
pandoc -f markdown -t org -o README.org README.md ;; convert from markdown to org
pandoc -f org TutorialKeyBindings.org -o TutorialKeyBindings.pdf
\end{verbatim}

\section{Bookmark}
\label{sec:org58b0169}
\begin{center}
\begin{tabular}{ll}
\hline
C-x r m & Create bookmark of current file or directory you are visiting\\
C-x r b & Go to a bookmark\\
C-x r l & List bookmark\\
\end{tabular}
\end{center}

\begin{center}
\begin{tabular}{ll}
\hline
d & Make item for delete\\
x & Remove all that are marked for d\\
r & Rename current item's title\\
s & Save changes\\
\end{tabular}
\end{center}

\begin{verbatim}
(setq bookmark-save-flag 1) ; everytime bookmark is changed, automatically save it
\end{verbatim}

\section{Prelude}
\label{sec:orgbaa703f}
\subsection{Install prelude}
\label{sec:org6fb2a20}
; Example src code\\
\begin{verbatim}
curl -L https://git.io/epre | sh
\end{verbatim}
\subsection{Other modules}
\label{sec:org6ea51b5}
(require 'prelude-helm)\\
(require 'prelude-key-chord)\\

\begin{verbatim}
; writing function
(defun my-some-function()
  (interactive)
  (function1 param)
  (function2 param))
\end{verbatim}
\subsection{Global shortcuts}
\label{sec:org7b0270f}
\begin{center}
\begin{tabular}{ll}
Keybinding & Description\\
\hline
C-x $\backslash$ & \texttt{align-regexp}\\
C-+ & Increase font size(\texttt{text-scale-increase}).\\
C-- & Decrease font size(\texttt{text-scale-decrease}).\\
C-x O & Go back to previous window (the inverse of \texttt{other-window} (\texttt{C-x o})).\\
C-\^{} & Join two lines into one(\texttt{crux-top-join-line}).\\
C-x p & Start \texttt{proced} (manage processes from Emacs; works only in Linux).\\
C-x m & Start \texttt{eshell}.\\
C-x M-m & Start your default shell.\\
C-x C-m & Alias for \texttt{M-x}.\\
M-X & Like \texttt{M-x} but limited to commands that are relevant to the active major mode.\\
C-h A & Run \texttt{apropos} (search in all Emacs symbols).\\
C-h C-m & Display key bindings of current major mode and descriptions of every binding.\\
M-/ & Run \texttt{hippie-expand} (a replacement for the default \texttt{dabbrev-expand}).\\
C-x C-b & Open \texttt{ibuffer} (a replacement for the default \texttt{buffer-list}).\\
F11 & Make the window full screen.\\
F12 & Toggle the Emacs menu bar.\\
C-x g & Open Magit's status buffer.\\
C-x M-g & Open Magit's popup of popups.\\
M-Z & Zap up to char.\\
C-= & Run \texttt{expand-region} (incremental text selection).\\
C-a & Run \texttt{crux-move-beginning-of-line}. Read \href{http://emacsredux.com/blog/2013/05/22/smarter-navigation-to-the-beginning-of-a-line/}{this} for details.\\
\end{tabular}
\end{center}

\subsection{Prelude Mode}
\label{sec:org471fb6e}
\begin{center}
\begin{tabular}{ll}
Keybinding & Description\\
\hline
C-c o & Open the currently visited file with an external program.\\
C-c i & Search for a symbol, only for buffers that contain code\\
C-c g & Search in Google for the thing under point (or an interactive query).\\
C-c G & Search in GitHub for the thing under point (or an interactive query).\\
C-c y & Search in YouTube for the thing under point (or an interactive query).\\
C-c U & Search in Duckduckgo for the thing under point (or an interactive query).\\
C-S-RET or Super-o & Insert an empty line above the current line and indent it properly.\\
S-RET or M-o & Insert an empty line and indent it properly (as in most IDEs).\\
C-S-up or M-S-up & Move the current line or region up.\\
C-S-down or M-S-down & Move the current line or region down.\\
C-c n & Fix indentation in buffer and strip whitespace.\\
C-c f & Open recently visited file.\\
C-M-$\backslash$ & Indent region (if selected) or the entire buffer.\\
C-c u & Open a new buffer containing the contents of URL.\\
C-c e & Eval a bit of Emacs Lisp code and replace it with its result.\\
C-c s & Swap two active windows.\\
C-c D & Delete current file and buffer.\\
C-c d & Duplicate the current line (or region).\\
C-c M-d & Duplicate and comment the current line (or region).\\
C-c r & Rename the current buffer and its visiting file if any.\\
C-c t & Open a terminal emulator (\texttt{ansi-term}).\\
C-c k & Kill all open buffers except the one you're currently in.\\
C-c TAB & Indent and copy region to clipboard\\
C-c I & Open user's init file.\\
C-c S & Open shell's init file.\\
C-c . + & Increment integer at point. Default is +1.\\
C-c . - & Decrement integer at point. Default is -1.\\
C-c . * & Multiply integer at point. Default is *2.\\
C-c . / & Divide integer at point. Default is /2.\\
C-c . $\backslash$ & Modulo integer at point. Default is modulo 2.\\
C-c . \^{} & Power to the integer at point. Default is \^{}2.\\
C-c . < & Left-shift integer at point. Default is 1 position to the left.\\
C-c . > & Right-shift integer at point. Default is 1 position to the right.\\
C-c . \# & Convert integer at point to specified base. Default is 10.\\
C-c . \% & Replace integer at point with another specified integer.\\
C-c . ' & Perform arithmetic operations on integer at point. User specifies operator.\\
Super-r & Recent files\\
Super-j & Join lines\\
Super-k & Kill whole line\\
Super-m m & Magit status\\
Super-m l & Magit log\\
Super-m f & Magit file log\\
Super-m b & Magit blame mode\\
\end{tabular}
\end{center}

\section{Magit}
\label{sec:org300b090}
Getting started with Magit is really easy:\\
\begin{center}
\begin{tabular}{ll}
\hline
M-x magit-status or C-x g & to see git status, and in the status buffer\\
s & stage files\\
c c & commit (type the message then C-c C-c to actually commit)\\
b b & switch to another branch\\
P u & git push\\
F u & git pull\\
\end{tabular}
\end{center}


\subsection{Help projectile keys}
\label{sec:org2165359}
\begin{center}
\begin{tabular}{ll}
\hline
C-c p C-h or s-p C-h & \\
or s-p & and wait for a moment for key bindings for projectile\\
C-c h & helm\\
C-c h C-h & Key bindings in helm\\
\end{tabular}
\end{center}


\section{Dired Mode}
\label{sec:org38aa13f}
\begin{center}
\begin{tabular}{ll}
Keybinding & Description\\
\hline
Enter/o & Open a file\\
g & Refresh\\
\^{} & Visit Parent Directory\\
\end{tabular}
\end{center}

\begin{center}
\begin{tabular}{ll}
m & Mark file\\
u/U & Unmark/unmark all\\
\%m & Mark by regexp\\
\%g & Make files with content matching regexp\\
t & Toggle mark all\\
\end{tabular}
\end{center}

\begin{center}
\begin{tabular}{ll}
d/D & Flag for deletion/delete now\\
\%d & Flag by regexp\\
\end{tabular}
\end{center}

\begin{center}
\begin{tabular}{ll}
R & Rename/Move\\
C & Copy\\
i/+ & Create directory\\
\end{tabular}
\end{center}

Note: For more see: \href{dired-ref.pdf}{Dired Ref}\\

\section{Basic from tutorial}
\label{sec:orge4abc7f}
\subsection{Move}
\label{sec:orgf603350}
\begin{center}
\begin{tabular}{ll}
\hline
C-f & Move forward a character\\
C-b & Move backward a character\\
M-f & Move forward a word\\
M-b & Move backward a word\\
C-n & Move to next line\\
C-p & Move to previous line\\
C-a & Move to beginning of line\\
C-e & Move to end of line\\
M-a & Move back to beginning of sentence\\
M-e & Move forward to end of sentence\\
\end{tabular}
\end{center}

C-u 8 *  Insert * 8 times\\
\subsection{Delete}
\label{sec:org8525b03}
<DEL>        Delete the character just before the cursor\\
C-d   	     Delete the next character after the cursor\\

M-<DEL>      Kill the word immediately before the cursor\\
M-d	     Kill the next word after the cursor\\

C-k	     Kill from the cursor position to end of line\\
M-k	     Kill to the end of the current sentence\\
\subsection{Undo}
\label{sec:org992d502}
C-/          Undo\\
C-\_          Undo\\
C-x u        Undo\\
C-x z        Redo previous command\\
C-g C-/      Redo\\
C-S-/        Redo\\


\subsection{File}
\label{sec:orgdb56b72}
C-x C-f		Find file\\
C-x C-s		Save file\\
C-x s		Save some buffers\\
C-x C-b		List buffers\\
C-x b		Switch buffer\\
C-x C-c		Quit Emacs\\
C-x 1		Delete all but one window\\
C-x u		Undo\\


\subsection{Multiple windows}
\label{sec:orgb8f96f5}
C-M-v             Scroll other window\\
C-x o             Go to other window\\
C-x 1             Get rid of other windows\\
C-x 0             Get rid of current window\\
C-x 4 C-f         Open in other window\\
C-x 5 2           Open new frame\\
C-x 5 0           Remove selected frame\\

\subsection{Help}
\label{sec:orgd0ef535}
ESC ESC ESC       Get out of recersive mode\\
C-h c <cmd/char/seq>   Shows help on cmd. E.g C-h c C-p shows what C-p does\\
C-x C-c           Exit emacs (Only do this if you restart)\\

\section{Org mode}
\label{sec:org687ff94}
\begin{center}
\begin{tabular}{ll}
\hline
C-c C-l & Org insert link\\
C-c C-o & Org open at point\\
\hline
\end{tabular}
\end{center}

\subsection{Create a file with todo and add it to agenda}
\label{sec:org7cf8632}
Org mode:\\
\begin{center}
\begin{tabular}{ll}
\hline
C-c a & agenda\\
C-c & add document to the list of agenda files\\
C-c & remove document from the list of agenda files\\
C-c & add date\\
C-u C-c . – add time and date & \\
C-g – stop doing what you are trying to do, escape & \\
\end{tabular}
\end{center}

\subsection{Tasks}
\label{sec:org4248eaf}

\begin{center}
\begin{tabular}{ll}
C-c C-t & Task done\\
Link [[link][description]$\backslash$] & \\
Shift tab & Show all headings\\
C-c a t & Global tab list\\
C-c C-s & Org schedule. Go to a list and schedule it.\\
l & In agenda, turns on log display (finished task and finished dates)\\
\end{tabular}
\end{center}




Note: For more see: \href{orgcard.pdf}{Org mode Ref Card}\\

\section{Printing}
\label{sec:org3e35710}
\begin{center}
\begin{tabular}{ll}
\hline
lpq / lpstat & Show print queue\\
cancel -a & Cancel all print jobs\\
lpr <files> & Print files\\
\hline
\end{tabular}
\end{center}
\section{Tags}
\label{sec:org027395a}
\subsection{To Generate Tags}
\label{sec:orgdd04bab}
\begin{verbatim}
find . -type f -iname "*.[chS]" | xargs etags -a
find . -name '*.c' -exec etags -a {} \;
\end{verbatim}
\subsection{To Genereate Emacs Tags}
\label{sec:orga4d9bef}
\begin{verbatim}
ctags -e myfile.cpp
ctags -e -R .
ctags -e -R *.cpp *.hpp *.h
\end{verbatim}

\subsection{Basic commands}
\label{sec:orgcb04120}

\begin{center}
\begin{tabular}{ll}
\hline
Command Name & Action\\
\hline
M-. <RET> & Jump to the tag underneath the cursor\\
M-. <tag> <RET> & Search for a particular Tag\\
C-u M-. & Find the next definition for the last Tag\\
M-* & Pop back to where you previously invoked M-.\\
\hline
\end{tabular}
\end{center}

\section{Projectile}
\label{sec:org81cbc25}
\begin{center}
\begin{tabular}{ll}
\hline
Keys & Description\\
\hline
C-c p C-h & Projectile's key bindings\\
C-c h & Go to a projectile project (most important)\\
C-x j & Projectile dired (Ctrl - m = Go inside director)\\
\hline
C-x j & Projectile dired (Ctrl - m: Inside, v: down, \^{}: up)\\
\hline
\end{tabular}
\end{center}

\begin{verbatim}
ctags -e -R . : Generate emacs ctags
\end{verbatim}
\begin{center}
\begin{tabular}{ll}
\hline
M-. , M-. tag & Jump to tag or search a tag\\
C-u M-. & Find the next definition for the last tag\\
M-* & Pop back to where you previously invoked M-.\\
\hline
\end{tabular}
\end{center}
\end{document}
